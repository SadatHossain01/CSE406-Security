\newpage
\section{Source Code}
Wazuh source code is publicly available on github. There are 24 public repositories associated with Wazuh, each containing modules for the core back-end, search index, front-end, documentation etc. We currently focused our attention towards the front-end.

\subsection{Repositories}
Below are the four primary repositories associated with the Wazuh project:

\paragraph{Wazuh}
\begin{description}
    \item[Repository URL:] \url{https://github.com/wazuh/wazuh}
    \item[Description:] This repository contains the backend source code for Wazuh Managers and Agents written in C, C++ and Python.
\end{description}

\paragraph{Wazuh Dashboard}
\begin{description}
    \item[Repository URL:] \url{https://github.com/wazuh/wazuh-dashboard}
    \item[Description:] Wazuh dashboard is a fork of the OpenSearch Dashboards which incorporate changes to make it easier to use for Wazuh users. It doesn't provide any specific UI, rather it is the platform on which Wazuh web UI runs on.
\end{description}

\paragraph{Wazuh Dashboard Plugins}
\begin{description}
    \item[Repository URL:] \url{https://github.com/wazuh/wazuh-dashboard-plugins}
    \item[Description:] This repository contains a set of plugins for Wazuh dashboard. Essentially providing all the UI components used on the Wazuh Web app.
\end{description}

\paragraph{Wazuh Indexer}
\begin{description}
    \item[Repository URL:] \url{https://github.com/wazuh/wazuh-dashboard}
    \item[Description:] This repository contains a highly scalable, full-text search and analytics engine. This Wazuh central component indexes and stores alerts generated by the Wazuh server and provides near real-time data search and analytics capabilities.
\end{description}

\subsection{Compiling the Front-end from Source}
From the repository structure and descriptions, it was evident that \texttt{wazuh-dashboard-plugins} repository hosted all of the front-end source code.

We followed the contributor's guide and documentation to compile the repository and create a development environment for the front-end. The steps to recreate the environment is outlined below-

\begin{enumerate}
    \item Remove or disable standalone Docker Engine (if installed). Install \href{https://docs.docker.com/get-docker/}{Docker Desktop}.
    \item Configure the docker environment.
    \begin{minted}{bash}
docker network create devel
docker network create mon
docker plugin install grafana/loki-docker-driver:latest \
    --alias loki --grant-all-permissions
    \end{minted}
    \item Assign enough resources to Docker Desktop. At least -
    \begin{itemize}
        \item 8 GB of RAM
        \item 4 CPU Cores
    \end{itemize}
    \item Save the path to the \texttt{plugins} folder inside \texttt{wazuh-dashboard-plugins} repository code as an environment variable, by exporting this path on .bashrc, .zhsrc or similar.
    \begin{minted}{bash}
./bashrc
export WZ_HOME=~/code/wazuh-dashboard-plugins/plugins
    \end{minted}
    \item The Docker volumes will be created by the internal Docker user, making them read-only. Which will prevent us from modifying the source code while running the environment. To prevent this, a new group named \texttt{docker-desktop} and GUID 100999 needs to be created, then added to the user and the source code folder:
    \begin{minted}{bash}
sudo groupadd -g 100999 docker-desktop
sudo useradd -u 100999 -g 100999 -M docker-desktop
sudo chown -R $USER:docker-desktop $WZ_HOME
sudo chmod -R 774 $WZ_HOME
sudo usermod -aG docker-desktop $USER
    \end{minted}
    \item Clone the repository.
    \begin{minted}{bash}
git clone https://github.com/wazuh/wazuh-dashboard-plugins.git
cd wazuh-dashboard-plugins
    \end{minted}
    \item Checkout to tag \texttt{v4.7.2-2.8.0}, corresponding to \texttt{Wazuh v4.7.2} release with \texttt{OpenSearch Dashboards 2.8.0}.
    \begin{minted}{bash}
git checkout v4.7.2-2.8.0
    \end{minted}

    \item The \texttt{docker} folder inside the repository contains various docker images to create development and testing environments. We use the \texttt{osd-dev} environment.
    \begin{minted}{bash}
cd docker/osd-dev
    \end{minted}
    \item Use the \texttt{dev.sh} script to call \texttt{docker-compose} and spin up the containers required for the development environment.
    \begin{minted}{bash}
./dev.sh 2.8.0 2.8.0 $WZ_HOME/main up server 4.7.2
    \end{minted}
    where, 
    \begin{itemize}
        \item \texttt{os\_version=2.8.0} is the OpenSearch version
        \item \texttt{osd\_version=2.8.0} is the OpenSearch Dashboard version
        \item \texttt{os\_version=\$WZ\_HOME/main} is the path to the Wazuh Application source code
        \item \texttt{action=up} is the action to do (one of \texttt{up}, \texttt{down} or \texttt{stop}.
        \item \texttt{server} to create an environment with a Wazuh Server running.
        \item \texttt{server\_version} version of the Wazuh server.
    \end{itemize}
    \item Also, add a agent container with the command:
    \begin{minted}{bash}
docker run --name os-dev-280-agent-$(date +%s) \
    --network os-dev-2.8.0 \
    --label com.docker.compose.project=os-dev-280 \
    --env WAZUH_AGENT_VERSION=4.7.2 \
    -d ubuntu:20.04 bash -c 
    'apt update -y
    apt install -y curl lsb-release
    curl -so \wazuh-agent-${WAZUH_AGENT_VERSION}.deb \
        "https://packages.wazuh.com/4.x/apt/pool/main/w/wazuh-agent/"\
        "wazuh-agent_${WAZUH_AGENT_VERSION}-1_amd64.deb" \
        && WAZUH_MANAGER='wazuh.manager' WAZUH_AGENT_GROUP='default' \
        dpkg -i ./wazuh-agent-${WAZUH_AGENT_VERSION}.deb
    /etc/init.d/wazuh-agent start
    tail -f /var/ossec/logs/ossec.log'
    \end{minted}
    \item Attach a shell to the \texttt{os-dev-280-osd-1} docker container to go inside the development environment.
    \begin{minted}{bash}
docker exec -it os-dev-280-osd-1 /bin/bash
    \end{minted}
    \item Install the dependencies using:
    \begin{minted}{bash}
yarn install 
    \end{minted}
    \item Run the Web server on \texttt{https://0.0.0.0:5601/} using:
    \begin{minted}{bash}
yarn start --no-base-path 
    \end{minted}
\end{enumerate}
The server usually takes a few moments to load all the comments. Once it's loaded login using credentials \texttt{admin:admin}.